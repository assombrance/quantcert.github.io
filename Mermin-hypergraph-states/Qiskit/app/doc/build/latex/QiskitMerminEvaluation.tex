%% Generated by Sphinx.
\def\sphinxdocclass{report}
\documentclass[letterpaper,10pt,english]{sphinxmanual}
\ifdefined\pdfpxdimen
   \let\sphinxpxdimen\pdfpxdimen\else\newdimen\sphinxpxdimen
\fi \sphinxpxdimen=.75bp\relax

\PassOptionsToPackage{warn}{textcomp}
\usepackage[utf8]{inputenc}
\ifdefined\DeclareUnicodeCharacter
% support both utf8 and utf8x syntaxes
\edef\sphinxdqmaybe{\ifdefined\DeclareUnicodeCharacterAsOptional\string"\fi}
  \DeclareUnicodeCharacter{\sphinxdqmaybe00A0}{\nobreakspace}
  \DeclareUnicodeCharacter{\sphinxdqmaybe2500}{\sphinxunichar{2500}}
  \DeclareUnicodeCharacter{\sphinxdqmaybe2502}{\sphinxunichar{2502}}
  \DeclareUnicodeCharacter{\sphinxdqmaybe2514}{\sphinxunichar{2514}}
  \DeclareUnicodeCharacter{\sphinxdqmaybe251C}{\sphinxunichar{251C}}
  \DeclareUnicodeCharacter{\sphinxdqmaybe2572}{\textbackslash}
\fi
\usepackage{cmap}
\usepackage[T1]{fontenc}
\usepackage{amsmath,amssymb,amstext}
\usepackage{babel}
\usepackage{times}
\usepackage[Bjarne]{fncychap}
\usepackage{sphinx}

\fvset{fontsize=\small}
\usepackage{geometry}

% Include hyperref last.
\usepackage{hyperref}
% Fix anchor placement for figures with captions.
\usepackage{hypcap}% it must be loaded after hyperref.
% Set up styles of URL: it should be placed after hyperref.
\urlstyle{same}
\addto\captionsenglish{\renewcommand{\contentsname}{Contents:}}

\addto\captionsenglish{\renewcommand{\figurename}{Fig.\@ }}
\makeatletter
\def\fnum@figure{\figurename\thefigure{}}
\makeatother
\addto\captionsenglish{\renewcommand{\tablename}{Table }}
\makeatletter
\def\fnum@table{\tablename\thetable{}}
\makeatother
\addto\captionsenglish{\renewcommand{\literalblockname}{Listing}}

\addto\captionsenglish{\renewcommand{\literalblockcontinuedname}{continued from previous page}}
\addto\captionsenglish{\renewcommand{\literalblockcontinuesname}{continues on next page}}
\addto\captionsenglish{\renewcommand{\sphinxnonalphabeticalgroupname}{Non-alphabetical}}
\addto\captionsenglish{\renewcommand{\sphinxsymbolsname}{Symbols}}
\addto\captionsenglish{\renewcommand{\sphinxnumbersname}{Numbers}}

\addto\extrasenglish{\def\pageautorefname{page}}

\setcounter{tocdepth}{1}



\title{Qiskit Mermin Evaluation}
\date{Oct 01, 2020}
\release{}
\author{}
\newcommand{\sphinxlogo}{\vbox{}}
\renewcommand{\releasename}{}
\makeindex
\begin{document}

\pagestyle{empty}
\sphinxmaketitle
\pagestyle{plain}
\sphinxtableofcontents
\pagestyle{normal}
\phantomsection\label{\detokenize{index::doc}}



\chapter{Evaluation}
\label{\detokenize{evaluation:module-mermin_on_qiskit.evaluation}}\label{\detokenize{evaluation:evaluation}}\label{\detokenize{evaluation::doc}}\index{mermin\_on\_qiskit.evaluation (module)@\spxentry{mermin\_on\_qiskit.evaluation}\spxextra{module}}\index{evaluate\_monomial() (in module mermin\_on\_qiskit.evaluation)@\spxentry{evaluate\_monomial()}\spxextra{in module mermin\_on\_qiskit.evaluation}}

\begin{fulllineitems}
\phantomsection\label{\detokenize{evaluation:mermin_on_qiskit.evaluation.evaluate_monomial}}\pysiglinewithargsret{\sphinxcode{\sphinxupquote{mermin\_on\_qiskit.evaluation.}}\sphinxbfcode{\sphinxupquote{evaluate\_monomial}}}{\emph{n}, \emph{n\_measure}, \emph{circuit}, \emph{a\_a\_p\_coeffs}, \emph{shots}, \emph{is\_simulation=True}, \emph{monitor=False}, \emph{local=True}}{}~\begin{description}
\item[{Draws the circuit if there are some additions and runs it to get the }] \leavevmode
measurement of a monomial

\end{description}
\begin{quote}\begin{description}
\item[{Parameters}] \leavevmode\begin{itemize}
\item {} 
\sphinxstyleliteralstrong{\sphinxupquote{n}} (\sphinxstyleliteralemphasis{\sphinxupquote{int}}) \textendash{} The number of qubits.

\item {} 
\sphinxstyleliteralstrong{\sphinxupquote{n\_measure}} (\sphinxstyleliteralemphasis{\sphinxupquote{int}}) \textendash{} The measurement to be performed. Dictates whether a\_i 
or a’\_i is used on each wire.

\item {} 
\sphinxstyleliteralstrong{\sphinxupquote{circuit}} (\sphinxstyleliteralemphasis{\sphinxupquote{QuantumCircuit}}) \textendash{} The original quantum circuit.

\item {} 
\sphinxstyleliteralstrong{\sphinxupquote{a\_a\_p\_coeffs}} (\sphinxstyleliteralemphasis{\sphinxupquote{array}}\sphinxstyleliteralemphasis{\sphinxupquote{{[}}}\sphinxstyleliteralemphasis{\sphinxupquote{float}}\sphinxstyleliteralemphasis{\sphinxupquote{{]}}}) \textendash{} The coefficients of the matrices used to 
calculate Mermin operators.

\item {} 
\sphinxstyleliteralstrong{\sphinxupquote{shots}} (\sphinxstyleliteralemphasis{\sphinxupquote{int}}) \textendash{} The number of repetitions of each circuit. Default: 1024.

\item {} 
\sphinxstyleliteralstrong{\sphinxupquote{is\_simulation}} (\sphinxstyleliteralemphasis{\sphinxupquote{boolean}}) \textendash{} This determines if we are in a case of a local
test or a real IBM machine test.

\item {} 
\sphinxstyleliteralstrong{\sphinxupquote{monitor}} (\sphinxstyleliteralemphasis{\sphinxupquote{boolean}}) \textendash{} If true a monitor is attached to the job.

\item {} 
\sphinxstyleliteralstrong{\sphinxupquote{local}} (\sphinxstyleliteralemphasis{\sphinxupquote{boolean}}) \textendash{} If true, the job run on a local simulator.

\end{itemize}

\item[{Returns}] \leavevmode
float \textendash{} The result of the measurement probabilities on one 
monomial.

\end{description}\end{quote}

\end{fulllineitems}

\index{evaluate\_polynomial() (in module mermin\_on\_qiskit.evaluation)@\spxentry{evaluate\_polynomial()}\spxextra{in module mermin\_on\_qiskit.evaluation}}

\begin{fulllineitems}
\phantomsection\label{\detokenize{evaluation:mermin_on_qiskit.evaluation.evaluate_polynomial}}\pysiglinewithargsret{\sphinxcode{\sphinxupquote{mermin\_on\_qiskit.evaluation.}}\sphinxbfcode{\sphinxupquote{evaluate\_polynomial}}}{\emph{n}, \emph{circuit}, \emph{a\_a\_p\_coeffs}, \emph{shots=1024}, \emph{is\_simulation=True}, \emph{monitor=False}, \emph{local=True}}{}
Makes all the implementation and calculation
\begin{description}
\item[{Caution!}] \leavevmode{[}{]}
The IBMQ account must be loaded before the execution of this 
function if the variable is\_simulation is set to False.

\end{description}
\begin{quote}\begin{description}
\item[{Parameters}] \leavevmode\begin{itemize}
\item {} 
\sphinxstyleliteralstrong{\sphinxupquote{n}} (\sphinxstyleliteralemphasis{\sphinxupquote{int}}) \textendash{} The number of qubits.

\item {} 
\sphinxstyleliteralstrong{\sphinxupquote{circuit}} (\sphinxstyleliteralemphasis{\sphinxupquote{QuantumCircuit}}) \textendash{} The original quantum circuit.

\item {} 
\sphinxstyleliteralstrong{\sphinxupquote{a\_a\_p\_coeffs}} (\sphinxstyleliteralemphasis{\sphinxupquote{list}}\sphinxstyleliteralemphasis{\sphinxupquote{{[}}}\sphinxstyleliteralemphasis{\sphinxupquote{list}}\sphinxstyleliteralemphasis{\sphinxupquote{{[}}}\sphinxstyleliteralemphasis{\sphinxupquote{any}}\sphinxstyleliteralemphasis{\sphinxupquote{{]}}}\sphinxstyleliteralemphasis{\sphinxupquote{{]}}}) \textendash{} Lists of lists of elements as described 
above (packed coefficients).

\item {} 
\sphinxstyleliteralstrong{\sphinxupquote{shots}} (\sphinxstyleliteralemphasis{\sphinxupquote{int}}) \textendash{} The number of times that the measurements are made. This 
is only in case of a local test.

\item {} 
\sphinxstyleliteralstrong{\sphinxupquote{is\_simulation}} (\sphinxstyleliteralemphasis{\sphinxupquote{boolean}}) \textendash{} To specify if the codes are to run locally or 
on the IBM machine.

\item {} 
\sphinxstyleliteralstrong{\sphinxupquote{monitor}} (\sphinxstyleliteralemphasis{\sphinxupquote{boolean}}) \textendash{} If true a monitor is attached to the job.

\item {} 
\sphinxstyleliteralstrong{\sphinxupquote{local}} (\sphinxstyleliteralemphasis{\sphinxupquote{boolean}}) \textendash{} If true, the job run on a local simulator.

\end{itemize}

\item[{Returns}] \leavevmode
float \textendash{} The result of all the calculations.

\end{description}\end{quote}

\end{fulllineitems}

\index{measures\_exploitation() (in module mermin\_on\_qiskit.evaluation)@\spxentry{measures\_exploitation()}\spxextra{in module mermin\_on\_qiskit.evaluation}}

\begin{fulllineitems}
\phantomsection\label{\detokenize{evaluation:mermin_on_qiskit.evaluation.measures_exploitation}}\pysiglinewithargsret{\sphinxcode{\sphinxupquote{mermin\_on\_qiskit.evaluation.}}\sphinxbfcode{\sphinxupquote{measures\_exploitation}}}{\emph{measures\_dictionary}, \emph{shots}}{}
Calculates the measurements probabilities

For every possible cases (for example, with n = 2 : 00 01 10 11), the 
probability to get this combination when measuring is calculated.

In order to obtain this probabilities, we sum the values of the cases where 
the number of 1 in the measurement is even and when it’s then odd.

For example :
even\_results = values of 00 and 11 measurements
odd\_results = values of 01 and 10 measurements
\begin{quote}\begin{description}
\item[{Parameters}] \leavevmode\begin{itemize}
\item {} 
\sphinxstyleliteralstrong{\sphinxupquote{measures\_dictionary}} (\sphinxstyleliteralemphasis{\sphinxupquote{dict}}) \textendash{} the dictionary containing the measurements 
and their values.

\item {} 
\sphinxstyleliteralstrong{\sphinxupquote{shots}} (\sphinxstyleliteralemphasis{\sphinxupquote{int}}) \textendash{} the number of times that the measurements are made. This 
is only in case of a local test.

\end{itemize}

\item[{Returns}] \leavevmode
float \textendash{} The total probability of the dictionary measurement.

\end{description}\end{quote}

\end{fulllineitems}

\index{mermin\_IBM() (in module mermin\_on\_qiskit.evaluation)@\spxentry{mermin\_IBM()}\spxextra{in module mermin\_on\_qiskit.evaluation}}

\begin{fulllineitems}
\phantomsection\label{\detokenize{evaluation:mermin_on_qiskit.evaluation.mermin_IBM}}\pysiglinewithargsret{\sphinxcode{\sphinxupquote{mermin\_on\_qiskit.evaluation.}}\sphinxbfcode{\sphinxupquote{mermin\_IBM}}}{\emph{n}}{}~\begin{description}
\item[{Returns the Mermin polynomials under a vector form. This form helps to }] \leavevmode
form the corrects monomials that involves in every mermin evaluation.

\item[{Example :}] \leavevmode
In this case, the involving monomials are only the second one, the third 
one, the fith one and the last one because the others are equal to zero.
\textgreater{}\textgreater{}\textgreater{} mermin\_IBM(3)
{[}0.0, 0.5, 0.5, 0.0, 0.5, 0.0, 0.0, -0.5{]}

\end{description}
\begin{quote}\begin{description}
\item[{Parameters}] \leavevmode
\sphinxstyleliteralstrong{\sphinxupquote{n}} (\sphinxstyleliteralemphasis{\sphinxupquote{int}}) \textendash{} The number of qubits.

\item[{Returns}] \leavevmode
list(float) \textendash{} The list of numbers corresponding to the existence 
and the value each monomial.

\end{description}\end{quote}

\end{fulllineitems}



\chapter{Basis Change}
\label{\detokenize{basis_change:module-mermin_on_qiskit.basis_change}}\label{\detokenize{basis_change:basis-change}}\label{\detokenize{basis_change::doc}}\index{mermin\_on\_qiskit.basis\_change (module)@\spxentry{mermin\_on\_qiskit.basis\_change}\spxextra{module}}\index{U3\_gates\_placement() (in module mermin\_on\_qiskit.basis\_change)@\spxentry{U3\_gates\_placement()}\spxextra{in module mermin\_on\_qiskit.basis\_change}}

\begin{fulllineitems}
\phantomsection\label{\detokenize{basis_change:mermin_on_qiskit.basis_change.U3_gates_placement}}\pysiglinewithargsret{\sphinxcode{\sphinxupquote{mermin\_on\_qiskit.basis\_change.}}\sphinxbfcode{\sphinxupquote{U3\_gates\_placement}}}{\emph{n}, \emph{n\_measure}, \emph{a\_a\_p\_coeffs}, \emph{circuit}}{}
Places the U3 gates according to the mermin\_IBM monomial
\begin{quote}\begin{description}
\item[{Parameters}] \leavevmode\begin{itemize}
\item {} 
\sphinxstyleliteralstrong{\sphinxupquote{n}} (\sphinxstyleliteralemphasis{\sphinxupquote{int}}) \textendash{} the size of the register to be evaluated

\item {} 
\sphinxstyleliteralstrong{\sphinxupquote{n\_measure}} (\sphinxstyleliteralemphasis{\sphinxupquote{int}}) \textendash{} The measure to be performed. Dictates whether a\_i or
a’\_i is used on each wire

\item {} 
\sphinxstyleliteralstrong{\sphinxupquote{a\_a\_p\_coeffs}} (\sphinxstyleliteralemphasis{\sphinxupquote{list}}\sphinxstyleliteralemphasis{\sphinxupquote{{[}}}\sphinxstyleliteralemphasis{\sphinxupquote{list}}\sphinxstyleliteralemphasis{\sphinxupquote{{[}}}\sphinxstyleliteralemphasis{\sphinxupquote{real}}\sphinxstyleliteralemphasis{\sphinxupquote{{]}}}\sphinxstyleliteralemphasis{\sphinxupquote{{]}}}) \textendash{} Contains the list of coefficients for 
a\_i and a’\_i in the packed shape

\item {} 
\sphinxstyleliteralstrong{\sphinxupquote{circuit}} (\sphinxstyleliteralemphasis{\sphinxupquote{QuantumCircuit}}) \textendash{} The circuit on which the measures are
appended

\end{itemize}

\item[{Returns}] \leavevmode
None

\end{description}\end{quote}

\end{fulllineitems}

\index{convert\_in\_binary() (in module mermin\_on\_qiskit.basis\_change)@\spxentry{convert\_in\_binary()}\spxextra{in module mermin\_on\_qiskit.basis\_change}}

\begin{fulllineitems}
\phantomsection\label{\detokenize{basis_change:mermin_on_qiskit.basis_change.convert_in_binary}}\pysiglinewithargsret{\sphinxcode{\sphinxupquote{mermin\_on\_qiskit.basis\_change.}}\sphinxbfcode{\sphinxupquote{convert\_in\_binary}}}{\emph{number\_to\_convert}, \emph{number\_of\_bits=0}}{}
Converts an int into a string containing its bits
\begin{description}
\item[{Example :}] \leavevmode
\begin{sphinxVerbatim}[commandchars=\\\{\}]
\PYG{g+gp}{\PYGZgt{}\PYGZgt{}\PYGZgt{} }\PYG{n}{convert\PYGZus{}in\PYGZus{}binary}\PYG{p}{(}\PYG{l+m+mi}{5}\PYG{p}{,}\PYG{l+m+mi}{3}\PYG{p}{)}
\PYG{g+go}{101}
\PYG{g+gp}{\PYGZgt{}\PYGZgt{}\PYGZgt{} }\PYG{n}{convert\PYGZus{}in\PYGZus{}binary}\PYG{p}{(}\PYG{l+m+mi}{5}\PYG{p}{,}\PYG{l+m+mi}{5}\PYG{p}{)}
\PYG{g+go}{00101}
\end{sphinxVerbatim}

\end{description}
\begin{quote}\begin{description}
\item[{Parameters}] \leavevmode\begin{itemize}
\item {} 
\sphinxstyleliteralstrong{\sphinxupquote{number\_to\_convert}} (\sphinxstyleliteralemphasis{\sphinxupquote{int}}) \textendash{} The number that is going to be converted.

\item {} 
\sphinxstyleliteralstrong{\sphinxupquote{number\_of\_bits}} (\sphinxstyleliteralemphasis{\sphinxupquote{int}}) \textendash{} The number of bits required.

\end{itemize}

\item[{Returns}] \leavevmode
str \textendash{} The converted number.

\end{description}\end{quote}

\end{fulllineitems}

\index{mermin\_coeffs\_to\_U3\_coeffs() (in module mermin\_on\_qiskit.basis\_change)@\spxentry{mermin\_coeffs\_to\_U3\_coeffs()}\spxextra{in module mermin\_on\_qiskit.basis\_change}}

\begin{fulllineitems}
\phantomsection\label{\detokenize{basis_change:mermin_on_qiskit.basis_change.mermin_coeffs_to_U3_coeffs}}\pysiglinewithargsret{\sphinxcode{\sphinxupquote{mermin\_on\_qiskit.basis\_change.}}\sphinxbfcode{\sphinxupquote{mermin\_coeffs\_to\_U3\_coeffs}}}{\emph{x}, \emph{y}, \emph{z}}{}~\begin{description}
\item[{Generates the coefficients of the U3 gate from mermin coefficients:}] \leavevmode
x*X + y*Y + z*Z = U3(theta, phi, -phi-pi)

\end{description}
\begin{quote}\begin{description}
\item[{Parameters}] \leavevmode\begin{itemize}
\item {} 
\sphinxstyleliteralstrong{\sphinxupquote{x}} (\sphinxstyleliteralemphasis{\sphinxupquote{float}}) \textendash{} The coefficient alpha for the matrix X.

\item {} 
\sphinxstyleliteralstrong{\sphinxupquote{y}} (\sphinxstyleliteralemphasis{\sphinxupquote{float}}) \textendash{} The coefficient beta for the matrix Y.

\item {} 
\sphinxstyleliteralstrong{\sphinxupquote{z}} (\sphinxstyleliteralemphasis{\sphinxupquote{float}}) \textendash{} The coefficient gamma for the matrix Z.

\end{itemize}

\item[{Returns}] \leavevmode
(float, float) \textendash{} The two angles of U3 gate.

\end{description}\end{quote}

\end{fulllineitems}



\chapter{Run}
\label{\detokenize{run:module-mermin_on_qiskit.run}}\label{\detokenize{run:run}}\label{\detokenize{run::doc}}\index{mermin\_on\_qiskit.run (module)@\spxentry{mermin\_on\_qiskit.run}\spxextra{module}}\index{load\_IBMQ\_account() (in module mermin\_on\_qiskit.run)@\spxentry{load\_IBMQ\_account()}\spxextra{in module mermin\_on\_qiskit.run}}

\begin{fulllineitems}
\phantomsection\label{\detokenize{run:mermin_on_qiskit.run.load_IBMQ_account}}\pysiglinewithargsret{\sphinxcode{\sphinxupquote{mermin\_on\_qiskit.run.}}\sphinxbfcode{\sphinxupquote{load\_IBMQ\_account}}}{}{}
Loads the IMBQ account. If it fails a first time, the IBMQ token will be
prompted and the account loading will be attempted a second time. If it fails
a second time. Exits by letting the \(Error\) be raised.
\begin{description}
\item[{Raises:}] \leavevmode\begin{description}
\item[{IBMQAccountCredentialsInvalidFormat: If the default provider stored on}] \leavevmode
disk could not be parsed.

\item[{IBMQAccountCredentialsNotFound: If no IBM Quantum Experience credentials}] \leavevmode
can be found.

\item[{IBMQAccountMultipleCredentialsFound: If multiple IBM Quantum Experience}] \leavevmode
credentials are found.

\item[{IBMQAccountCredentialsInvalidUrl: If invalid IBM Quantum Experience}] \leavevmode
credentials are found.

\item[{IBMQProviderError: If the default provider stored on disk could not}] \leavevmode
be found.

\end{description}

\end{description}

\end{fulllineitems}

\index{runCircuit() (in module mermin\_on\_qiskit.run)@\spxentry{runCircuit()}\spxextra{in module mermin\_on\_qiskit.run}}

\begin{fulllineitems}
\phantomsection\label{\detokenize{run:mermin_on_qiskit.run.runCircuit}}\pysiglinewithargsret{\sphinxcode{\sphinxupquote{mermin\_on\_qiskit.run.}}\sphinxbfcode{\sphinxupquote{runCircuit}}}{\emph{qc}, \emph{simulation=True}, \emph{return\_count=True}, \emph{monitor=False}, \emph{local=True}, \emph{shots=1024}}{}
Runs the QuantumCircuit \(qc\) in IBM Quantum Experience.
\begin{quote}\begin{description}
\item[{Parameters}] \leavevmode\begin{itemize}
\item {} 
\sphinxstyleliteralstrong{\sphinxupquote{qc}} (\sphinxstyleliteralemphasis{\sphinxupquote{QuantumCircuit}}) \textendash{} Quantum circuit to be executed

\item {} 
\sphinxstyleliteralstrong{\sphinxupquote{simulation}} (\sphinxstyleliteralemphasis{\sphinxupquote{bool}}) \textendash{} If \(True\), the experience runs on a simulator, which 
substantially faster than on a quantum processor (due to the demand on
those). Otherwise, runs on one of the quantum processors.

\item {} 
\sphinxstyleliteralstrong{\sphinxupquote{return\_count}} (\sphinxstyleliteralemphasis{\sphinxupquote{bool}}) \textendash{} If the circuit contains measures, and return\_count
is set to \(True\), then the count of the result will be returned, otherwise,
the result will be directly returned.

\item {} 
\sphinxstyleliteralstrong{\sphinxupquote{monitor}} (\sphinxstyleliteralemphasis{\sphinxupquote{bool}}) \textendash{} If \(True\), a \(job_monitor\) will be displayed after the
job is submitted.

\end{itemize}

\end{description}\end{quote}

TODO : add local and shots docs
\begin{quote}\begin{description}
\item[{Returns}] \leavevmode
dict{[}str:int{]} or Result \textendash{} Depending on return\_count, \(runCircuit\)
either returns the result (of type Result) of the run or the count of this
result, which would be the equivalent of calling \(result.get_counts()\).

\end{description}\end{quote}

\end{fulllineitems}



\chapter{Coefficients shapes}
\label{\detokenize{coefficients_shapes:module-mermin_on_qiskit.coefficients_shapes}}\label{\detokenize{coefficients_shapes:coefficients-shapes}}\label{\detokenize{coefficients_shapes::doc}}\index{mermin\_on\_qiskit.coefficients\_shapes (module)@\spxentry{mermin\_on\_qiskit.coefficients\_shapes}\spxextra{module}}
There are three format used for the algorithms:

1. A flat list of coefficients, organized as such: 
\([x1,y1,z1, x2,y2, ..., xn,yn,zn, x'1,y'1,z'1, x'2,y'2, ..., x'n,y'n,z'n]\).
This format is called the \sphinxstyleemphasis{unpacked coefficients} and is used for the QFT 
optimization.
2. A list of coefficients grouped by families of operators:
\([[[x1,y1,z1], [x2,y2, ..., [xn,yn,zn]], [[x'1,y'1,z'1], [x'2,y'2, ..., [x'n,y'n,z'n]]]\)
in other words, you have the whole \(a\) family and the the whole \(a'\) family,
and in a family, you have \(a1\), \(a2\), and so on… Each \(a\) is described by
it’s three coefficients: \(x\), \(y\) and \(z\).
This format is called \sphinxstyleemphasis{packed coefficients} and is used to easily manipulate
coefficients.
3. A list of coefficients grouped by operator:
\([[x1,y1,z1], [x'1,y'1,z'1], [x2,y2,z2], [x'2,y'2,z'2], ...]\)
in other words, the list is formed as such: \([a1, a'1, a2, a'2, ...]\)
This format was previously used for evaluation in Qiskit, allowing for a
simpler data flow. But it has the inconvenient of being less true to the 
maths behind all this so it has been dropped. 
This format is called \sphinxstyleemphasis{mixed}

With the functions of this module, one may switch between \sphinxstyleemphasis{1.} and \sphinxstyleemphasis{2.} and 
between \sphinxstyleemphasis{2.} and \sphinxstyleemphasis{3.}, allowing tho switch freely between the three formats.
\index{coefficients\_format\_mixed\_to\_packed() (in module mermin\_on\_qiskit.coefficients\_shapes)@\spxentry{coefficients\_format\_mixed\_to\_packed()}\spxextra{in module mermin\_on\_qiskit.coefficients\_shapes}}

\begin{fulllineitems}
\phantomsection\label{\detokenize{coefficients_shapes:mermin_on_qiskit.coefficients_shapes.coefficients_format_mixed_to_packed}}\pysiglinewithargsret{\sphinxcode{\sphinxupquote{mermin\_on\_qiskit.coefficients\_shapes.}}\sphinxbfcode{\sphinxupquote{coefficients\_format\_mixed\_to\_packed}}}{\emph{\_a\_a\_prime\_coeffs}}{}
Format the coefficients in the shape previously used for evaluation
\begin{description}
\item[{Example:}] \leavevmode
\begin{sphinxVerbatim}[commandchars=\\\{\}]
\PYG{g+gp}{\PYGZgt{}\PYGZgt{}\PYGZgt{} }\PYG{n}{coefficients\PYGZus{}format\PYGZus{}mixed\PYGZus{}to\PYGZus{}packed}\PYG{p}{(}\PYG{p}{[}\PYG{p}{[}\PYG{l+m+mi}{1}\PYG{p}{,} \PYG{l+m+mi}{2}\PYG{p}{,} \PYG{l+m+mi}{3}\PYG{p}{]}\PYG{p}{,} \PYG{p}{[}\PYG{l+m+mi}{7}\PYG{p}{,} \PYG{l+m+mi}{8}\PYG{p}{,} \PYG{l+m+mi}{9}\PYG{p}{]}\PYG{p}{,} \PYG{p}{[}\PYG{l+m+mi}{4}\PYG{p}{,} \PYG{l+m+mi}{5}\PYG{p}{,} \PYG{l+m+mi}{6}\PYG{p}{]}\PYG{p}{,} \PYG{p}{[}\PYG{l+m+mi}{10}\PYG{p}{,} \PYG{l+m+mi}{11}\PYG{p}{,} \PYG{l+m+mi}{12}\PYG{p}{]}\PYG{p}{]}\PYG{p}{)}                   
\PYG{g+go}{([[1, 2, 3], [4, 5, 6]], [[7, 8, 9], [10, 11, 12]])}
\end{sphinxVerbatim}

\end{description}
\begin{quote}\begin{description}
\item[{Parameters}] \leavevmode
\sphinxstyleliteralstrong{\sphinxupquote{\_a\_a\_prime\_coeffs}} (\sphinxstyleliteralemphasis{\sphinxupquote{list}}\sphinxstyleliteralemphasis{\sphinxupquote{{[}}}\sphinxstyleliteralemphasis{\sphinxupquote{list}}\sphinxstyleliteralemphasis{\sphinxupquote{{[}}}\sphinxstyleliteralemphasis{\sphinxupquote{any}}\sphinxstyleliteralemphasis{\sphinxupquote{{]}}}\sphinxstyleliteralemphasis{\sphinxupquote{{]}}}) \textendash{} List of lists of elements as 
described above (mixed coefficients).

\item[{Returns}] \leavevmode
tuple(list{[}list{[}any{]}{]}) \textendash{} Tuple of list of list of elements (packed
coefficients).

\end{description}\end{quote}

\end{fulllineitems}

\index{coefficients\_format\_packed\_to\_mixed() (in module mermin\_on\_qiskit.coefficients\_shapes)@\spxentry{coefficients\_format\_packed\_to\_mixed()}\spxextra{in module mermin\_on\_qiskit.coefficients\_shapes}}

\begin{fulllineitems}
\phantomsection\label{\detokenize{coefficients_shapes:mermin_on_qiskit.coefficients_shapes.coefficients_format_packed_to_mixed}}\pysiglinewithargsret{\sphinxcode{\sphinxupquote{mermin\_on\_qiskit.coefficients\_shapes.}}\sphinxbfcode{\sphinxupquote{coefficients\_format\_packed\_to\_mixed}}}{\emph{\_a\_coeffs}, \emph{\_a\_prime\_coeffs}}{}
Format the coefficients in the shape now used for evaluation
\begin{description}
\item[{Example:}] \leavevmode
\begin{sphinxVerbatim}[commandchars=\\\{\}]
\PYG{g+gp}{\PYGZgt{}\PYGZgt{}\PYGZgt{} }\PYG{n}{coefficients\PYGZus{}format\PYGZus{}packed\PYGZus{}to\PYGZus{}mixed}\PYG{p}{(}\PYG{p}{[}\PYG{p}{[}\PYG{l+m+mi}{1}\PYG{p}{,}\PYG{l+m+mi}{2}\PYG{p}{,}\PYG{l+m+mi}{3}\PYG{p}{]}\PYG{p}{,}\PYG{p}{[}\PYG{l+m+mi}{4}\PYG{p}{,}\PYG{l+m+mi}{5}\PYG{p}{,}\PYG{l+m+mi}{6}\PYG{p}{]}\PYG{p}{]}\PYG{p}{,} \PYG{p}{[}\PYG{p}{[}\PYG{l+m+mi}{7}\PYG{p}{,}\PYG{l+m+mi}{8}\PYG{p}{,}\PYG{l+m+mi}{9}\PYG{p}{]}\PYG{p}{,}\PYG{p}{[}\PYG{l+m+mi}{10}\PYG{p}{,}\PYG{l+m+mi}{11}\PYG{p}{,}\PYG{l+m+mi}{12}\PYG{p}{]}\PYG{p}{]}\PYG{p}{)}                           
\PYG{g+go}{[[1, 2, 3], [7, 8, 9], [4, 5, 6], [10, 11, 12]]}
\end{sphinxVerbatim}

\end{description}
\begin{quote}\begin{description}
\item[{Parameters}] \leavevmode
\sphinxstyleliteralstrong{\sphinxupquote{\_a\_coeffs}}\sphinxstyleliteralstrong{\sphinxupquote{, }}\sphinxstyleliteralstrong{\sphinxupquote{\_a\_prime\_coeffs}} (\sphinxstyleliteralemphasis{\sphinxupquote{list}}\sphinxstyleliteralemphasis{\sphinxupquote{{[}}}\sphinxstyleliteralemphasis{\sphinxupquote{list}}\sphinxstyleliteralemphasis{\sphinxupquote{{[}}}\sphinxstyleliteralemphasis{\sphinxupquote{any}}\sphinxstyleliteralemphasis{\sphinxupquote{{]}}}\sphinxstyleliteralemphasis{\sphinxupquote{{]}}}) \textendash{} Lists of lists of 
elements as described above (packed coefficients).

\item[{Returns}] \leavevmode
list{[}any{]} \textendash{} List of list of elements (mixed coefficients).

\end{description}\end{quote}

\end{fulllineitems}

\index{coefficients\_format\_packed\_to\_unpacked() (in module mermin\_on\_qiskit.coefficients\_shapes)@\spxentry{coefficients\_format\_packed\_to\_unpacked()}\spxextra{in module mermin\_on\_qiskit.coefficients\_shapes}}

\begin{fulllineitems}
\phantomsection\label{\detokenize{coefficients_shapes:mermin_on_qiskit.coefficients_shapes.coefficients_format_packed_to_unpacked}}\pysiglinewithargsret{\sphinxcode{\sphinxupquote{mermin\_on\_qiskit.coefficients\_shapes.}}\sphinxbfcode{\sphinxupquote{coefficients\_format\_packed\_to\_unpacked}}}{\emph{\_a\_coeffs}, \emph{\_a\_prime\_coeffs}}{}
Unpacks two lists of lists of three elements to one list of elements
\begin{description}
\item[{Example:}] \leavevmode
\begin{sphinxVerbatim}[commandchars=\\\{\}]
\PYG{g+gp}{\PYGZgt{}\PYGZgt{}\PYGZgt{} }\PYG{n}{coefficients\PYGZus{}format\PYGZus{}packed\PYGZus{}to\PYGZus{}unpacked}\PYG{p}{(}\PYG{p}{[}\PYG{p}{[}\PYG{l+m+mi}{1}\PYG{p}{,}\PYG{l+m+mi}{2}\PYG{p}{,}\PYG{l+m+mi}{3}\PYG{p}{]}\PYG{p}{,}\PYG{p}{[}\PYG{l+m+mi}{4}\PYG{p}{,}\PYG{l+m+mi}{5}\PYG{p}{,}\PYG{l+m+mi}{6}\PYG{p}{]}\PYG{p}{]}\PYG{p}{,}\PYG{p}{[}\PYG{p}{[}\PYG{l+m+mi}{7}\PYG{p}{,}\PYG{l+m+mi}{8}\PYG{p}{,}\PYG{l+m+mi}{9}\PYG{p}{]}\PYG{p}{,}\PYG{p}{[}\PYG{l+m+mi}{10}\PYG{p}{,}\PYG{l+m+mi}{11}\PYG{p}{,}\PYG{l+m+mi}{12}\PYG{p}{]}\PYG{p}{]}\PYG{p}{)}
\PYG{g+go}{[1,2,3,4,5,6,7,8,9,10,11,12]}
\end{sphinxVerbatim}

\end{description}
\begin{quote}\begin{description}
\item[{Parameters}] \leavevmode
\sphinxstyleliteralstrong{\sphinxupquote{\_a\_coeffs}}\sphinxstyleliteralstrong{\sphinxupquote{, }}\sphinxstyleliteralstrong{\sphinxupquote{\_a\_prime\_coeffs}} (\sphinxstyleliteralemphasis{\sphinxupquote{list}}\sphinxstyleliteralemphasis{\sphinxupquote{{[}}}\sphinxstyleliteralemphasis{\sphinxupquote{list}}\sphinxstyleliteralemphasis{\sphinxupquote{{[}}}\sphinxstyleliteralemphasis{\sphinxupquote{any}}\sphinxstyleliteralemphasis{\sphinxupquote{{]}}}\sphinxstyleliteralemphasis{\sphinxupquote{{]}}}) \textendash{} Lists of lists of 
elements as described above (packed coefficients).

\item[{Returns}] \leavevmode
list{[}any{]} \textendash{} List of elements (unpacked coefficients).

\end{description}\end{quote}

\end{fulllineitems}

\index{coefficients\_format\_unpacked\_to\_packed() (in module mermin\_on\_qiskit.coefficients\_shapes)@\spxentry{coefficients\_format\_unpacked\_to\_packed()}\spxextra{in module mermin\_on\_qiskit.coefficients\_shapes}}

\begin{fulllineitems}
\phantomsection\label{\detokenize{coefficients_shapes:mermin_on_qiskit.coefficients_shapes.coefficients_format_unpacked_to_packed}}\pysiglinewithargsret{\sphinxcode{\sphinxupquote{mermin\_on\_qiskit.coefficients\_shapes.}}\sphinxbfcode{\sphinxupquote{coefficients\_format\_unpacked\_to\_packed}}}{\emph{\_a\_a\_prime\_coeffs}}{}
Packs a list of elements in two lists of lists of three elements
\begin{description}
\item[{Example:}] \leavevmode
\begin{sphinxVerbatim}[commandchars=\\\{\}]
\PYG{g+gp}{\PYGZgt{}\PYGZgt{}\PYGZgt{} }\PYG{n}{coefficients\PYGZus{}format\PYGZus{}unpacked\PYGZus{}to\PYGZus{}packed}\PYG{p}{(}\PYG{p}{[}\PYG{l+m+mi}{1}\PYG{p}{,}\PYG{l+m+mi}{2}\PYG{p}{,}\PYG{l+m+mi}{3}\PYG{p}{,}\PYG{l+m+mi}{4}\PYG{p}{,}\PYG{l+m+mi}{5}\PYG{p}{,}\PYG{l+m+mi}{6}\PYG{p}{,}\PYG{l+m+mi}{7}\PYG{p}{,}\PYG{l+m+mi}{8}\PYG{p}{,}\PYG{l+m+mi}{9}\PYG{p}{,}\PYG{l+m+mi}{10}\PYG{p}{,}\PYG{l+m+mi}{11}\PYG{p}{,}\PYG{l+m+mi}{12}\PYG{p}{]}\PYG{p}{)}
\PYG{g+go}{([[1,2,3],[4,5,6]],[[7,8,9],[10,11,12]])}
\end{sphinxVerbatim}

\end{description}
\begin{quote}\begin{description}
\item[{Parameters}] \leavevmode
\sphinxstyleliteralstrong{\sphinxupquote{\_a\_a\_prime\_coeffs}} (\sphinxstyleliteralemphasis{\sphinxupquote{list}}\sphinxstyleliteralemphasis{\sphinxupquote{{[}}}\sphinxstyleliteralemphasis{\sphinxupquote{any}}\sphinxstyleliteralemphasis{\sphinxupquote{{]}}}) \textendash{} List of elements (unpacked coefficients).

\item[{Returns}] \leavevmode
tuple{[}list{[}list{[}any{]}{]}{]} \textendash{} Lists of lists of elements as described 
above (packed coefficients).

\end{description}\end{quote}

\end{fulllineitems}



\chapter{QFT main}
\label{\detokenize{QFT:module-mermin_on_qiskit.QFT}}\label{\detokenize{QFT:qft-main}}\label{\detokenize{QFT::doc}}\index{mermin\_on\_qiskit.QFT (module)@\spxentry{mermin\_on\_qiskit.QFT}\spxextra{module}}
This module builds the QFT in Qiskit and runs it
\index{QFT\_lenght() (in module mermin\_on\_qiskit.QFT)@\spxentry{QFT\_lenght()}\spxextra{in module mermin\_on\_qiskit.QFT}}

\begin{fulllineitems}
\phantomsection\label{\detokenize{QFT:mermin_on_qiskit.QFT.QFT_lenght}}\pysiglinewithargsret{\sphinxcode{\sphinxupquote{mermin\_on\_qiskit.QFT.}}\sphinxbfcode{\sphinxupquote{QFT\_lenght}}}{\emph{nWires}}{}
\end{fulllineitems}

\index{all\_QFT\_circuits() (in module mermin\_on\_qiskit.QFT)@\spxentry{all\_QFT\_circuits()}\spxextra{in module mermin\_on\_qiskit.QFT}}

\begin{fulllineitems}
\phantomsection\label{\detokenize{QFT:mermin_on_qiskit.QFT.all_QFT_circuits}}\pysiglinewithargsret{\sphinxcode{\sphinxupquote{mermin\_on\_qiskit.QFT.}}\sphinxbfcode{\sphinxupquote{all\_QFT\_circuits}}}{\emph{nWires}}{}
\end{fulllineitems}

\index{build\_QFT\_0\_to\_k() (in module mermin\_on\_qiskit.QFT)@\spxentry{build\_QFT\_0\_to\_k()}\spxextra{in module mermin\_on\_qiskit.QFT}}

\begin{fulllineitems}
\phantomsection\label{\detokenize{QFT:mermin_on_qiskit.QFT.build_QFT_0_to_k}}\pysiglinewithargsret{\sphinxcode{\sphinxupquote{mermin\_on\_qiskit.QFT.}}\sphinxbfcode{\sphinxupquote{build\_QFT\_0\_to\_k}}}{\emph{nWires}, \emph{k}, \emph{measure=False}}{}
Builds the QFT on nWires wires up to the \(k^{th}\) state generates.
Note that the whole QFT can be generated in Qiskit using the \(QFT\) method
\begin{quote}\begin{description}
\item[{Parameters}] \leavevmode\begin{itemize}
\item {} 
\sphinxstyleliteralstrong{\sphinxupquote{nWires}} (\sphinxstyleliteralemphasis{\sphinxupquote{int}}) \textendash{} number of wires

\item {} 
\sphinxstyleliteralstrong{\sphinxupquote{k}} (\sphinxstyleliteralemphasis{\sphinxupquote{int}}) \textendash{} number of gates in the output

\end{itemize}

\item[{Returns}] \leavevmode
QuantumCirctuit \textendash{} \(k^{th}\) first gates of the QFT circuit

\end{description}\end{quote}

\end{fulllineitems}

\index{get\_coef\_from\_optimization\_file() (in module mermin\_on\_qiskit.QFT)@\spxentry{get\_coef\_from\_optimization\_file()}\spxextra{in module mermin\_on\_qiskit.QFT}}

\begin{fulllineitems}
\phantomsection\label{\detokenize{QFT:mermin_on_qiskit.QFT.get_coef_from_optimization_file}}\pysiglinewithargsret{\sphinxcode{\sphinxupquote{mermin\_on\_qiskit.QFT.}}\sphinxbfcode{\sphinxupquote{get\_coef\_from\_optimization\_file}}}{\emph{filename}, \emph{iteration}, \emph{evaluation=False}}{}~\begin{description}
\item[{The file fed in this function must be a csv file with one of columns being}] \leavevmode
named “iteration”, an other one being named “coefficients” and if the
\(evaluation\) parameter is set to \(True\) a column named “intricationValue”.
The “iteration” column must contain integers, the “coefficients” column
must contain tuples of real numbers (the mermin coefficients) and the
“intricationValue” column must contain real numbers.

\item[{Example:}] \leavevmode
\begin{sphinxVerbatim}[commandchars=\\\{\}]
\PYG{g+gp}{\PYGZgt{}\PYGZgt{}\PYGZgt{} }\PYG{n}{get\PYGZus{}coef\PYGZus{}from\PYGZus{}optimization\PYGZus{}file}\PYG{p}{(}\PYG{l+s+s2}{\PYGZdq{}}\PYG{l+s+s2}{../Sage/grover\PYGZus{}pac/examples/qft\PYGZus{}optimization/1\PYGZhy{}1\PYGZhy{}4.csv}\PYG{l+s+s2}{\PYGZdq{}}\PYG{p}{,}\PYG{l+m+mi}{2}\PYG{p}{)}
\PYG{g+go}{([(\PYGZhy{}0.197738971530022, \PYGZhy{}0.00983193840670331, 0.980205403028067), }
\PYG{g+go}{    (0.892812904656093, \PYGZhy{}0.035586934469795, 0.449019695976237), }
\PYG{g+go}{    (\PYGZhy{}0.892282320991669, \PYGZhy{}0.00788653439204609, \PYGZhy{}0.451408974457756), }
\PYG{g+go}{    (0.982839628418978, 0.012341254672589, \PYGZhy{}0.184048793102134)], }
\PYG{g+go}{  [(0.430894968126063, \PYGZhy{}0.0211632981654613, 0.902153890006799), }
\PYG{g+go}{    (\PYGZhy{}0.984337747324624, 0.0105235779205133, 0.175978559772592), }
\PYG{g+go}{    (0.984221659519729, \PYGZhy{}0.0118117927364157, \PYGZhy{}0.176545196719091), }
\PYG{g+go}{    (0.883187500168151, 0.0083188846423974, 0.468946303647911)])}
\end{sphinxVerbatim}

\end{description}
\begin{quote}\begin{description}
\item[{Parameters}] \leavevmode\begin{itemize}
\item {} 
\sphinxstyleliteralstrong{\sphinxupquote{filename}} (\sphinxstyleliteralemphasis{\sphinxupquote{str}}) \textendash{} Name of the CSV file containing the information about the
Mermin coefficients.

\item {} 
\sphinxstyleliteralstrong{\sphinxupquote{iteration}} (\sphinxstyleliteralemphasis{\sphinxupquote{int}}) \textendash{} Designates the line from which the data must be
retrieved.

\item {} 
\sphinxstyleliteralstrong{\sphinxupquote{intricationValue}} (\sphinxstyleliteralemphasis{\sphinxupquote{bool}}) \textendash{} If \(True\), the evaluation will be returned as 
well as the coefficients.

\end{itemize}

\item[{Returns}] \leavevmode
tuple{[}list{[}tuple{[}real{]}{]}{]}, real(optional) \textendash{} The mermin coefficients 
previously optimized in packed shape, eventually with the optimum computed
with these coefficients.

\end{description}\end{quote}

\end{fulllineitems}

\index{periodic\_state() (in module mermin\_on\_qiskit.QFT)@\spxentry{periodic\_state()}\spxextra{in module mermin\_on\_qiskit.QFT}}

\begin{fulllineitems}
\phantomsection\label{\detokenize{QFT:mermin_on_qiskit.QFT.periodic_state}}\pysiglinewithargsret{\sphinxcode{\sphinxupquote{mermin\_on\_qiskit.QFT.}}\sphinxbfcode{\sphinxupquote{periodic\_state}}}{\emph{l}, \emph{r}, \emph{nWires}}{}
Returns the periodic state \(|\varphi^{l,r}>\) of size \(2^{nWires}\). We have:
\begin{quote}

\(|\varphi^{l,r}> = \sum_{i=0}^{A-1}|l+ir>/sqrt(A)\) with
\(A = floor((2^{nWires}-l)/r)+1\)

In this definition, \sphinxcode{\sphinxupquote{l}} is the shift of the state, and \sphinxcode{\sphinxupquote{r}} is the period 
of the state.
\end{quote}
\begin{description}
\item[{Example:}] \leavevmode
Since
\(|\varphi^{1,5}> = (|1>+|6>+|11>)/sqrt(3)=(|0001>+|0110>+|1011>)/sqrt(3)\),

\begin{sphinxVerbatim}[commandchars=\\\{\}]
\PYG{g+gp}{\PYGZgt{}\PYGZgt{}\PYGZgt{} }\PYG{n}{periodic\PYGZus{}state}\PYG{p}{(}\PYG{l+m+mi}{1}\PYG{p}{,}\PYG{l+m+mi}{5}\PYG{p}{,}\PYG{l+m+mi}{4}\PYG{p}{)}
\PYG{g+go}{(0, 1/3*sqrt(3), 0, 0, 0, 0, 1/3*sqrt(3), 0, 0, 0, 0, 1/3*sqrt(3), 0, 0, 0, 0)}
\end{sphinxVerbatim}

\end{description}
\begin{quote}\begin{description}
\item[{Parameters}] \leavevmode\begin{itemize}
\item {} 
\sphinxstyleliteralstrong{\sphinxupquote{l}} (\sphinxstyleliteralemphasis{\sphinxupquote{int}}) \textendash{} The shift of the state.

\item {} 
\sphinxstyleliteralstrong{\sphinxupquote{r}} (\sphinxstyleliteralemphasis{\sphinxupquote{int}}) \textendash{} The period of the state.

\item {} 
\sphinxstyleliteralstrong{\sphinxupquote{nWires}} (\sphinxstyleliteralemphasis{\sphinxupquote{int}}) \textendash{} The size of the system (number of qubits).

\end{itemize}

\item[{Returns}] \leavevmode
vector \textendash{} The state defined by \sphinxcode{\sphinxupquote{l}}, \sphinxcode{\sphinxupquote{r}} and \sphinxcode{\sphinxupquote{nWires}} according
to the definition given above.

\end{description}\end{quote}

\end{fulllineitems}



\chapter{Hypergraphstates}
\label{\detokenize{hypergraphstates:module-mermin_on_qiskit.hypergraphstates}}\label{\detokenize{hypergraphstates:hypergraphstates}}\label{\detokenize{hypergraphstates::doc}}\index{mermin\_on\_qiskit.hypergraphstates (module)@\spxentry{mermin\_on\_qiskit.hypergraphstates}\spxextra{module}}\index{circuit\_creation() (in module mermin\_on\_qiskit.hypergraphstates)@\spxentry{circuit\_creation()}\spxextra{in module mermin\_on\_qiskit.hypergraphstates}}

\begin{fulllineitems}
\phantomsection\label{\detokenize{hypergraphstates:mermin_on_qiskit.hypergraphstates.circuit_creation}}\pysiglinewithargsret{\sphinxcode{\sphinxupquote{mermin\_on\_qiskit.hypergraphstates.}}\sphinxbfcode{\sphinxupquote{circuit\_creation}}}{\emph{n}, \emph{hyperedges}}{}
Creates an empty circuit with the number of qubits required.
\begin{quote}\begin{description}
\item[{Parameters}] \leavevmode\begin{itemize}
\item {} 
\sphinxstyleliteralstrong{\sphinxupquote{n}} (\sphinxstyleliteralemphasis{\sphinxupquote{int}}) \textendash{} The number of qubits of which depends the number of wires to 
create.

\item {} 
\sphinxstyleliteralstrong{\sphinxupquote{hyperedges}} (\sphinxstyleliteralemphasis{\sphinxupquote{list}}\sphinxstyleliteralemphasis{\sphinxupquote{{[}}}\sphinxstyleliteralemphasis{\sphinxupquote{list}}\sphinxstyleliteralemphasis{\sphinxupquote{{[}}}\sphinxstyleliteralemphasis{\sphinxupquote{int}}\sphinxstyleliteralemphasis{\sphinxupquote{{]}}}\sphinxstyleliteralemphasis{\sphinxupquote{{]}}}) \textendash{} A list containing the lists of the 
vertices which are linked by an hyperedge.

\end{itemize}

\item[{Returns}] \leavevmode
QuantumCircuit \textendash{} A circuit with the required number of quantum 
wires and classical wires.

\end{description}\end{quote}

\end{fulllineitems}

\index{circuit\_initialisation() (in module mermin\_on\_qiskit.hypergraphstates)@\spxentry{circuit\_initialisation()}\spxextra{in module mermin\_on\_qiskit.hypergraphstates}}

\begin{fulllineitems}
\phantomsection\label{\detokenize{hypergraphstates:mermin_on_qiskit.hypergraphstates.circuit_initialisation}}\pysiglinewithargsret{\sphinxcode{\sphinxupquote{mermin\_on\_qiskit.hypergraphstates.}}\sphinxbfcode{\sphinxupquote{circuit\_initialisation}}}{\emph{n}, \emph{hyperedges}}{}~\begin{description}
\item[{Creates an empty circuit with the number of qubits required and places }] \leavevmode
the circuit in the initial state before adding gates for calculations.
Places and Hadamard gate on every main (non additional) qubits wire.
This is needed in order to place the qubits in a \(|+>\) state which is
\((|0> + |1>) / sqrt(2)\).

\end{description}
\begin{quote}\begin{description}
\item[{Parameters}] \leavevmode\begin{itemize}
\item {} 
\sphinxstyleliteralstrong{\sphinxupquote{n}} (\sphinxstyleliteralemphasis{\sphinxupquote{int}}) \textendash{} The number of qubits of which depends the number of wires to 
create.

\item {} 
\sphinxstyleliteralstrong{\sphinxupquote{hyperedges}} (\sphinxstyleliteralemphasis{\sphinxupquote{list}}\sphinxstyleliteralemphasis{\sphinxupquote{{[}}}\sphinxstyleliteralemphasis{\sphinxupquote{list}}\sphinxstyleliteralemphasis{\sphinxupquote{{[}}}\sphinxstyleliteralemphasis{\sphinxupquote{int}}\sphinxstyleliteralemphasis{\sphinxupquote{{]}}}\sphinxstyleliteralemphasis{\sphinxupquote{{]}}}) \textendash{} A list containing the lists of the 
vertices which are linked by an hyperedge.

\end{itemize}

\item[{Returns}] \leavevmode
QuantumCircuit \textendash{} The created and initialized circuit.

\end{description}\end{quote}

\end{fulllineitems}

\index{edges\_layout() (in module mermin\_on\_qiskit.hypergraphstates)@\spxentry{edges\_layout()}\spxextra{in module mermin\_on\_qiskit.hypergraphstates}}

\begin{fulllineitems}
\phantomsection\label{\detokenize{hypergraphstates:mermin_on_qiskit.hypergraphstates.edges_layout}}\pysiglinewithargsret{\sphinxcode{\sphinxupquote{mermin\_on\_qiskit.hypergraphstates.}}\sphinxbfcode{\sphinxupquote{edges\_layout}}}{\emph{n}, \emph{hyperedges}, \emph{circuit}}{}~\begin{description}
\item[{Disposes all the gates corresponding to the edges.}] \leavevmode
In fact, for a two-vertices edge, there not much to do, as for a 
three-vertices edge, too.
For more an edge of than three vertices, things are a little different. 
First, Toffoli gates are used to link qubits
two by two. The target qubit here is an auxiliary qubit.
Then, the last link is a simple CZ.
But all the Toffoli gates that we put create an entanglement which is 
removed by replacing exactly the same gates again after the CZ.

\end{description}
\begin{quote}\begin{description}
\item[{Parameters}] \leavevmode\begin{itemize}
\item {} 
\sphinxstyleliteralstrong{\sphinxupquote{n}} (\sphinxstyleliteralemphasis{\sphinxupquote{int}}) \textendash{} The number of qubits.

\item {} 
\sphinxstyleliteralstrong{\sphinxupquote{hyperedges}} (\sphinxstyleliteralemphasis{\sphinxupquote{list}}\sphinxstyleliteralemphasis{\sphinxupquote{{[}}}\sphinxstyleliteralemphasis{\sphinxupquote{list}}\sphinxstyleliteralemphasis{\sphinxupquote{{[}}}\sphinxstyleliteralemphasis{\sphinxupquote{int}}\sphinxstyleliteralemphasis{\sphinxupquote{{]}}}) \textendash{} The list of the vertices which are linked
by an hyperedge.

\item {} 
\sphinxstyleliteralstrong{\sphinxupquote{circuit}} (\sphinxstyleliteralemphasis{\sphinxupquote{QuantumCircuit}}) \textendash{} The circuit that will be modified.

\end{itemize}

\item[{Returns}] \leavevmode
None

\end{description}\end{quote}

\end{fulllineitems}



\chapter{Hypergraphstates used for optimization}
\label{\detokenize{hypergraphstates-opti:module-mermin_on_qiskit.hypergraphstates_optimization.hypergraphstates}}\label{\detokenize{hypergraphstates-opti:hypergraphstates-used-for-optimization}}\label{\detokenize{hypergraphstates-opti::doc}}\index{mermin\_on\_qiskit.hypergraphstates\_optimization.hypergraphstates (module)@\spxentry{mermin\_on\_qiskit.hypergraphstates\_optimization.hypergraphstates}\spxextra{module}}\index{convert\_in\_binary() (in module mermin\_on\_qiskit.hypergraphstates\_optimization.hypergraphstates)@\spxentry{convert\_in\_binary()}\spxextra{in module mermin\_on\_qiskit.hypergraphstates\_optimization.hypergraphstates}}

\begin{fulllineitems}
\phantomsection\label{\detokenize{hypergraphstates-opti:mermin_on_qiskit.hypergraphstates_optimization.hypergraphstates.convert_in_binary}}\pysiglinewithargsret{\sphinxcode{\sphinxupquote{mermin\_on\_qiskit.hypergraphstates\_optimization.hypergraphstates.}}\sphinxbfcode{\sphinxupquote{convert\_in\_binary}}}{\emph{number\_to\_convert}, \emph{number\_of\_bits=0}}{}
Converts an int into a string containing its bits.
\begin{description}
\item[{Example:}] \leavevmode
\begin{sphinxVerbatim}[commandchars=\\\{\}]
\PYG{g+gp}{\PYGZgt{}\PYGZgt{}\PYGZgt{} }\PYG{n}{convert\PYGZus{}in\PYGZus{}binary}\PYG{p}{(}\PYG{l+m+mi}{5}\PYG{p}{,}\PYG{l+m+mi}{3}\PYG{p}{)}
\PYG{g+go}{101}
\PYG{g+gp}{\PYGZgt{}\PYGZgt{}\PYGZgt{} }\PYG{n}{convert\PYGZus{}in\PYGZus{}binary}\PYG{p}{(}\PYG{l+m+mi}{5}\PYG{p}{,}\PYG{l+m+mi}{3}\PYG{p}{)}
\PYG{g+go}{00101}
\end{sphinxVerbatim}

\end{description}
\begin{quote}\begin{description}
\item[{Parameters}] \leavevmode\begin{itemize}
\item {} 
\sphinxstyleliteralstrong{\sphinxupquote{number\_to\_convert}} (\sphinxstyleliteralemphasis{\sphinxupquote{int}}) \textendash{} the number that is going to be converted.

\item {} 
\sphinxstyleliteralstrong{\sphinxupquote{number\_of\_bits}} (\sphinxstyleliteralemphasis{\sphinxupquote{int}}) \textendash{} the number of bits required.

\end{itemize}

\item[{Returns}] \leavevmode
String number : The converted number.

\end{description}\end{quote}

\end{fulllineitems}

\index{corresponding\_state\_determination() (in module mermin\_on\_qiskit.hypergraphstates\_optimization.hypergraphstates)@\spxentry{corresponding\_state\_determination()}\spxextra{in module mermin\_on\_qiskit.hypergraphstates\_optimization.hypergraphstates}}

\begin{fulllineitems}
\phantomsection\label{\detokenize{hypergraphstates-opti:mermin_on_qiskit.hypergraphstates_optimization.hypergraphstates.corresponding_state_determination}}\pysiglinewithargsret{\sphinxcode{\sphinxupquote{mermin\_on\_qiskit.hypergraphstates\_optimization.hypergraphstates.}}\sphinxbfcode{\sphinxupquote{corresponding\_state\_determination}}}{\emph{n}, \emph{state}, \emph{hyperedges}}{}~\begin{description}
\item[{Determines the states corresponding to an hyperedge. That is the states }] \leavevmode
where the vertices linked by the hyperedge are equal to 1.

\end{description}
\begin{quote}\begin{description}
\item[{Parameters}] \leavevmode\begin{itemize}
\item {} 
\sphinxstyleliteralstrong{\sphinxupquote{n}} (\sphinxstyleliteralemphasis{\sphinxupquote{int}}) \textendash{} The number of qubits.

\item {} 
\sphinxstyleliteralstrong{\sphinxupquote{state}} (\sphinxstyleliteralemphasis{\sphinxupquote{list}}\sphinxstyleliteralemphasis{\sphinxupquote{(}}\sphinxstyleliteralemphasis{\sphinxupquote{int}}\sphinxstyleliteralemphasis{\sphinxupquote{)}}) \textendash{} The state of the vector.

\item {} 
\sphinxstyleliteralstrong{\sphinxupquote{hyperedges}} (\sphinxstyleliteralemphasis{\sphinxupquote{list}}\sphinxstyleliteralemphasis{\sphinxupquote{(}}\sphinxstyleliteralemphasis{\sphinxupquote{list}}\sphinxstyleliteralemphasis{\sphinxupquote{(}}\sphinxstyleliteralemphasis{\sphinxupquote{int}}\sphinxstyleliteralemphasis{\sphinxupquote{)}}\sphinxstyleliteralemphasis{\sphinxupquote{)}}) \textendash{} a list containing the lists of the 
vertices which are linked by an hyperedge.

\end{itemize}

\item[{Param}] \leavevmode
boolean \textendash{} True if the state is in an hyperedge.

\end{description}\end{quote}

\end{fulllineitems}

\index{hyperedges\_computation() (in module mermin\_on\_qiskit.hypergraphstates\_optimization.hypergraphstates)@\spxentry{hyperedges\_computation()}\spxextra{in module mermin\_on\_qiskit.hypergraphstates\_optimization.hypergraphstates}}

\begin{fulllineitems}
\phantomsection\label{\detokenize{hypergraphstates-opti:mermin_on_qiskit.hypergraphstates_optimization.hypergraphstates.hyperedges_computation}}\pysiglinewithargsret{\sphinxcode{\sphinxupquote{mermin\_on\_qiskit.hypergraphstates\_optimization.hypergraphstates.}}\sphinxbfcode{\sphinxupquote{hyperedges\_computation}}}{\emph{n}, \emph{state\_vector}, \emph{hyperedges}}{}~\begin{description}
\item[{Puts the phases in the right places. Wherever there is an edge or an }] \leavevmode
hyperedge between some vertices, a minus sign is put where those 
vertices are all in state 1.

\item[{Example:}] \leavevmode
\begin{sphinxVerbatim}[commandchars=\\\{\}]
\PYG{g+gp}{\PYGZgt{}\PYGZgt{}\PYGZgt{} }\PYG{n}{hyperedges\PYGZus{}computation}\PYG{p}{(}\PYG{l+m+mi}{2}\PYG{p}{,} \PYG{p}{[}\PYG{l+m+mf}{0.5}\PYG{p}{,} \PYG{l+m+mf}{0.5}\PYG{p}{,} \PYG{l+m+mf}{0.5}\PYG{p}{,} \PYG{l+m+mf}{0.5}\PYG{p}{]}\PYG{p}{,} \PYG{p}{[}\PYG{p}{[}\PYG{l+m+mi}{0}\PYG{p}{,}\PYG{l+m+mi}{1}\PYG{p}{]}\PYG{p}{]}\PYG{p}{)}
\PYG{g+go}{[0.5, 0.5, 0.5, \PYGZhy{}0.5]}
\end{sphinxVerbatim}

\end{description}
\begin{quote}\begin{description}
\item[{Parameters}] \leavevmode\begin{itemize}
\item {} 
\sphinxstyleliteralstrong{\sphinxupquote{n}} (\sphinxstyleliteralemphasis{\sphinxupquote{int}}) \textendash{} The number of qubits.

\item {} 
\sphinxstyleliteralstrong{\sphinxupquote{state\_vector}} (\sphinxstyleliteralemphasis{\sphinxupquote{list}}\sphinxstyleliteralemphasis{\sphinxupquote{(}}\sphinxstyleliteralemphasis{\sphinxupquote{int}}\sphinxstyleliteralemphasis{\sphinxupquote{)}}) \textendash{} The initialized state vector.

\item {} 
\sphinxstyleliteralstrong{\sphinxupquote{hyperedges}} (\sphinxstyleliteralemphasis{\sphinxupquote{list}}\sphinxstyleliteralemphasis{\sphinxupquote{{[}}}\sphinxstyleliteralemphasis{\sphinxupquote{list}}\sphinxstyleliteralemphasis{\sphinxupquote{{[}}}\sphinxstyleliteralemphasis{\sphinxupquote{int}}\sphinxstyleliteralemphasis{\sphinxupquote{{]}}}) \textendash{} a list containing the lists of the 
vertices which are linked by an hyperedge.

\end{itemize}

\item[{Returns}] \leavevmode
list(int) \textendash{} The correct state vector.

\end{description}\end{quote}

\end{fulllineitems}

\index{putting\_in\_list() (in module mermin\_on\_qiskit.hypergraphstates\_optimization.hypergraphstates)@\spxentry{putting\_in\_list()}\spxextra{in module mermin\_on\_qiskit.hypergraphstates\_optimization.hypergraphstates}}

\begin{fulllineitems}
\phantomsection\label{\detokenize{hypergraphstates-opti:mermin_on_qiskit.hypergraphstates_optimization.hypergraphstates.putting_in_list}}\pysiglinewithargsret{\sphinxcode{\sphinxupquote{mermin\_on\_qiskit.hypergraphstates\_optimization.hypergraphstates.}}\sphinxbfcode{\sphinxupquote{putting\_in\_list}}}{\emph{number}}{}
Puts every figure of the number in a list.
\begin{description}
\item[{Example:}] \leavevmode
\begin{sphinxVerbatim}[commandchars=\\\{\}]
\PYG{g+gp}{\PYGZgt{}\PYGZgt{}\PYGZgt{} }\PYG{n}{putting\PYGZus{}in\PYGZus{}list}\PYG{p}{(}\PYG{l+m+mi}{000}\PYG{p}{)}
\PYG{g+go}{ [0, 0, 0]}
\end{sphinxVerbatim}

\end{description}
\begin{quote}\begin{description}
\item[{Parameters}] \leavevmode
\sphinxstyleliteralstrong{\sphinxupquote{number}} (\sphinxstyleliteralemphasis{\sphinxupquote{int}}) \textendash{} The number to split.

\item[{Returns}] \leavevmode
list(int) \textendash{} The list with every digit of the number in a case.

\end{description}\end{quote}

\end{fulllineitems}

\index{state\_vector\_initialisation() (in module mermin\_on\_qiskit.hypergraphstates\_optimization.hypergraphstates)@\spxentry{state\_vector\_initialisation()}\spxextra{in module mermin\_on\_qiskit.hypergraphstates\_optimization.hypergraphstates}}

\begin{fulllineitems}
\phantomsection\label{\detokenize{hypergraphstates-opti:mermin_on_qiskit.hypergraphstates_optimization.hypergraphstates.state_vector_initialisation}}\pysiglinewithargsret{\sphinxcode{\sphinxupquote{mermin\_on\_qiskit.hypergraphstates\_optimization.hypergraphstates.}}\sphinxbfcode{\sphinxupquote{state\_vector\_initialisation}}}{\emph{n}}{}~\begin{description}
\item[{Initializes the state vector; which is to create an array with the size }] \leavevmode
of 2 to the power of n. Every number in the array is equal to 1 over the
square root of 2 to the power of n.

\item[{Example:}] \leavevmode
\begin{sphinxVerbatim}[commandchars=\\\{\}]
\PYG{g+gp}{\PYGZgt{}\PYGZgt{}\PYGZgt{} }\PYG{n}{state\PYGZus{}vector\PYGZus{}initialisation}\PYG{p}{(}\PYG{l+m+mi}{3}\PYG{p}{)}
\PYG{g+go}{[0.35355339 0.35355339 0.35355339 0.35355339 0.35355339 0.35355339}
\PYG{g+go}{ 0.35355339 0.35355339]}
\end{sphinxVerbatim}

\end{description}
\begin{quote}\begin{description}
\item[{Parameters}] \leavevmode
\sphinxstyleliteralstrong{\sphinxupquote{n}} (\sphinxstyleliteralemphasis{\sphinxupquote{int}}) \textendash{} The number of qubits.

\item[{Returns}] \leavevmode
list(int) \textendash{} The initialized state vector.

\end{description}\end{quote}

\end{fulllineitems}

\index{states\_formation() (in module mermin\_on\_qiskit.hypergraphstates\_optimization.hypergraphstates)@\spxentry{states\_formation()}\spxextra{in module mermin\_on\_qiskit.hypergraphstates\_optimization.hypergraphstates}}

\begin{fulllineitems}
\phantomsection\label{\detokenize{hypergraphstates-opti:mermin_on_qiskit.hypergraphstates_optimization.hypergraphstates.states_formation}}\pysiglinewithargsret{\sphinxcode{\sphinxupquote{mermin\_on\_qiskit.hypergraphstates\_optimization.hypergraphstates.}}\sphinxbfcode{\sphinxupquote{states\_formation}}}{\emph{n}}{}
Calculates every state for n qubits.
\begin{description}
\item[{Example:}] \leavevmode
\begin{sphinxVerbatim}[commandchars=\\\{\}]
\PYG{g+gp}{\PYGZgt{}\PYGZgt{}\PYGZgt{} }\PYG{n}{states\PYGZus{}formation}\PYG{p}{(}\PYG{l+m+mi}{3}\PYG{p}{)}
\PYG{g+go}{ [[0, 0, 0], [0, 0, 1], [0, 1, 0], [0, 1, 1], [1, 0, 0], [1, 0, 1], [1, 1, 0], [1, 1, 1]]}
\end{sphinxVerbatim}

\end{description}
\begin{quote}\begin{description}
\item[{Parameters}] \leavevmode
\sphinxstyleliteralstrong{\sphinxupquote{n}} (\sphinxstyleliteralemphasis{\sphinxupquote{int}}) \textendash{} The number of qubits.

\item[{Returns}] \leavevmode
list(list(int)) \textendash{} A list of all the possible states of the qubits 
which are also contained in a list.

\end{description}\end{quote}

\end{fulllineitems}



\chapter{Mermin polynomials used for optimization}
\label{\detokenize{mermin_polynomials-opti:module-mermin_on_qiskit.hypergraphstates_optimization.mermin_polynomials}}\label{\detokenize{mermin_polynomials-opti:mermin-polynomials-used-for-optimization}}\label{\detokenize{mermin_polynomials-opti::doc}}\index{mermin\_on\_qiskit.hypergraphstates\_optimization.mermin\_polynomials (module)@\spxentry{mermin\_on\_qiskit.hypergraphstates\_optimization.mermin\_polynomials}\spxextra{module}}\index{a\_matrix() (in module mermin\_on\_qiskit.hypergraphstates\_optimization.mermin\_polynomials)@\spxentry{a\_matrix()}\spxextra{in module mermin\_on\_qiskit.hypergraphstates\_optimization.mermin\_polynomials}}

\begin{fulllineitems}
\phantomsection\label{\detokenize{mermin_polynomials-opti:mermin_on_qiskit.hypergraphstates_optimization.mermin_polynomials.a_matrix}}\pysiglinewithargsret{\sphinxcode{\sphinxupquote{mermin\_on\_qiskit.hypergraphstates\_optimization.mermin\_polynomials.}}\sphinxbfcode{\sphinxupquote{a\_matrix}}}{\emph{n}, \emph{t}}{}
Constitutes the matrix a of the Mermin polynomial.
\begin{quote}\begin{description}
\item[{Parameters}] \leavevmode\begin{itemize}
\item {} 
\sphinxstyleliteralstrong{\sphinxupquote{n}} (\sphinxstyleliteralemphasis{\sphinxupquote{int}}) \textendash{} The number of qubits.

\item {} 
\sphinxstyleliteralstrong{\sphinxupquote{t}} (\sphinxstyleliteralemphasis{\sphinxupquote{np.array}}\sphinxstyleliteralemphasis{\sphinxupquote{(}}\sphinxstyleliteralemphasis{\sphinxupquote{list}}\sphinxstyleliteralemphasis{\sphinxupquote{(}}\sphinxstyleliteralemphasis{\sphinxupquote{float}}\sphinxstyleliteralemphasis{\sphinxupquote{)}}\sphinxstyleliteralemphasis{\sphinxupquote{)}}) \textendash{} The table of coefficients.

\end{itemize}

\item[{Returns}] \leavevmode
np.array(complex) \textendash{} The matrice a of mn.

\end{description}\end{quote}

\end{fulllineitems}

\index{a\_prime\_matrix() (in module mermin\_on\_qiskit.hypergraphstates\_optimization.mermin\_polynomials)@\spxentry{a\_prime\_matrix()}\spxextra{in module mermin\_on\_qiskit.hypergraphstates\_optimization.mermin\_polynomials}}

\begin{fulllineitems}
\phantomsection\label{\detokenize{mermin_polynomials-opti:mermin_on_qiskit.hypergraphstates_optimization.mermin_polynomials.a_prime_matrix}}\pysiglinewithargsret{\sphinxcode{\sphinxupquote{mermin\_on\_qiskit.hypergraphstates\_optimization.mermin\_polynomials.}}\sphinxbfcode{\sphinxupquote{a\_prime\_matrix}}}{\emph{n}, \emph{t}}{}
Constitutes the matrix a’ of the Mermin polynomial.
\begin{quote}\begin{description}
\item[{Parameters}] \leavevmode\begin{itemize}
\item {} 
\sphinxstyleliteralstrong{\sphinxupquote{n}} (\sphinxstyleliteralemphasis{\sphinxupquote{int}}) \textendash{} The number of qubits.

\item {} 
\sphinxstyleliteralstrong{\sphinxupquote{t}} (\sphinxstyleliteralemphasis{\sphinxupquote{np.array}}\sphinxstyleliteralemphasis{\sphinxupquote{(}}\sphinxstyleliteralemphasis{\sphinxupquote{list}}\sphinxstyleliteralemphasis{\sphinxupquote{(}}\sphinxstyleliteralemphasis{\sphinxupquote{float}}\sphinxstyleliteralemphasis{\sphinxupquote{)}}\sphinxstyleliteralemphasis{\sphinxupquote{)}}) \textendash{} The table of coefficients.

\end{itemize}

\item[{Returns}] \leavevmode
np.array(complex) \textendash{} The matrice a’ of mn.

\end{description}\end{quote}

\end{fulllineitems}

\index{first\_coefficients\_generation() (in module mermin\_on\_qiskit.hypergraphstates\_optimization.mermin\_polynomials)@\spxentry{first\_coefficients\_generation()}\spxextra{in module mermin\_on\_qiskit.hypergraphstates\_optimization.mermin\_polynomials}}

\begin{fulllineitems}
\phantomsection\label{\detokenize{mermin_polynomials-opti:mermin_on_qiskit.hypergraphstates_optimization.mermin_polynomials.first_coefficients_generation}}\pysiglinewithargsret{\sphinxcode{\sphinxupquote{mermin\_on\_qiskit.hypergraphstates\_optimization.mermin\_polynomials.}}\sphinxbfcode{\sphinxupquote{first\_coefficients\_generation}}}{\emph{n}}{}
Generates the very first coefficients for the calculation of MU.
\begin{description}
\item[{Example :}] \leavevmode
\begin{sphinxVerbatim}[commandchars=\\\{\}]
\PYG{g+gp}{\PYGZgt{}\PYGZgt{}\PYGZgt{} }\PYG{n}{first\PYGZus{}coefficients\PYGZus{}generation}\PYG{p}{(}\PYG{l+m+mi}{2}\PYG{p}{)}
\PYG{g+go}{[[ 0.24006446 \PYGZhy{}0.97020025 0.03287126]}
\PYG{g+go}{ [0.72092088 \PYGZhy{}0.59054414  0.36267162]}
\PYG{g+go}{ [\PYGZhy{}0.76022821 0.64723032 \PYGZhy{}0.056089]}
\PYG{g+go}{ [\PYGZhy{}0.0278048  0.48298397 \PYGZhy{}0.87518763]]}
\end{sphinxVerbatim}

\end{description}
\begin{quote}\begin{description}
\item[{Parameters}] \leavevmode
\sphinxstyleliteralstrong{\sphinxupquote{n}} (\sphinxstyleliteralemphasis{\sphinxupquote{int}}) \textendash{} The number of qubits.

\item[{Returns}] \leavevmode
np.array(list(float)) \textendash{} The table of list of the coefficient 
taken randomly.

\end{description}\end{quote}

\end{fulllineitems}

\index{mermin() (in module mermin\_on\_qiskit.hypergraphstates\_optimization.mermin\_polynomials)@\spxentry{mermin()}\spxextra{in module mermin\_on\_qiskit.hypergraphstates\_optimization.mermin\_polynomials}}

\begin{fulllineitems}
\phantomsection\label{\detokenize{mermin_polynomials-opti:mermin_on_qiskit.hypergraphstates_optimization.mermin_polynomials.mermin}}\pysiglinewithargsret{\sphinxcode{\sphinxupquote{mermin\_on\_qiskit.hypergraphstates\_optimization.mermin\_polynomials.}}\sphinxbfcode{\sphinxupquote{mermin}}}{\emph{n}, \emph{t}}{}
Calculates the Mermin polynomial \(mn\).
\begin{quote}\begin{description}
\item[{Parameters}] \leavevmode\begin{itemize}
\item {} 
\sphinxstyleliteralstrong{\sphinxupquote{n}} (\sphinxstyleliteralemphasis{\sphinxupquote{int}}) \textendash{} The number of qubits.

\item {} 
\sphinxstyleliteralstrong{\sphinxupquote{t}} (\sphinxstyleliteralemphasis{\sphinxupquote{np.array}}\sphinxstyleliteralemphasis{\sphinxupquote{(}}\sphinxstyleliteralemphasis{\sphinxupquote{list}}\sphinxstyleliteralemphasis{\sphinxupquote{(}}\sphinxstyleliteralemphasis{\sphinxupquote{float}}\sphinxstyleliteralemphasis{\sphinxupquote{)}}\sphinxstyleliteralemphasis{\sphinxupquote{)}}) \textendash{} The table of coefficients.

\end{itemize}

\item[{Returns}] \leavevmode
np.array(complex) \textendash{} The Mermin polynomial \(mn\).

\end{description}\end{quote}

\end{fulllineitems}

\index{mermin\_prime() (in module mermin\_on\_qiskit.hypergraphstates\_optimization.mermin\_polynomials)@\spxentry{mermin\_prime()}\spxextra{in module mermin\_on\_qiskit.hypergraphstates\_optimization.mermin\_polynomials}}

\begin{fulllineitems}
\phantomsection\label{\detokenize{mermin_polynomials-opti:mermin_on_qiskit.hypergraphstates_optimization.mermin_polynomials.mermin_prime}}\pysiglinewithargsret{\sphinxcode{\sphinxupquote{mermin\_on\_qiskit.hypergraphstates\_optimization.mermin\_polynomials.}}\sphinxbfcode{\sphinxupquote{mermin\_prime}}}{\emph{n}, \emph{t}}{}
Calculates the Mermin polynomial \(mn'\).
\begin{quote}\begin{description}
\item[{Parameters}] \leavevmode\begin{itemize}
\item {} 
\sphinxstyleliteralstrong{\sphinxupquote{n}} (\sphinxstyleliteralemphasis{\sphinxupquote{int}}) \textendash{} The number of qubits

\item {} 
\sphinxstyleliteralstrong{\sphinxupquote{t}} (\sphinxstyleliteralemphasis{\sphinxupquote{np.array}}\sphinxstyleliteralemphasis{\sphinxupquote{(}}\sphinxstyleliteralemphasis{\sphinxupquote{list}}\sphinxstyleliteralemphasis{\sphinxupquote{(}}\sphinxstyleliteralemphasis{\sphinxupquote{float}}\sphinxstyleliteralemphasis{\sphinxupquote{)}}\sphinxstyleliteralemphasis{\sphinxupquote{)}}) \textendash{} The table of coefficients

\end{itemize}

\item[{Returns}] \leavevmode
np.array(complex) \textendash{} The Mermin polynomial \(mn'\)

\end{description}\end{quote}

\end{fulllineitems}

\index{mu\_calculation() (in module mermin\_on\_qiskit.hypergraphstates\_optimization.mermin\_polynomials)@\spxentry{mu\_calculation()}\spxextra{in module mermin\_on\_qiskit.hypergraphstates\_optimization.mermin\_polynomials}}

\begin{fulllineitems}
\phantomsection\label{\detokenize{mermin_polynomials-opti:mermin_on_qiskit.hypergraphstates_optimization.mermin_polynomials.mu_calculation}}\pysiglinewithargsret{\sphinxcode{\sphinxupquote{mermin\_on\_qiskit.hypergraphstates\_optimization.mermin\_polynomials.}}\sphinxbfcode{\sphinxupquote{mu\_calculation}}}{\emph{mn}, \emph{mn\_prime}, \emph{vector}, \emph{type\_of\_mu}}{}~\begin{description}
\item[{Calculates MU, the value of the calculation of the vector with the }] \leavevmode
mermin polynomial.

\end{description}
\begin{quote}\begin{description}
\item[{Parameters}] \leavevmode\begin{itemize}
\item {} 
\sphinxstyleliteralstrong{\sphinxupquote{mn}} (\sphinxstyleliteralemphasis{\sphinxupquote{np.array}}\sphinxstyleliteralemphasis{\sphinxupquote{(}}\sphinxstyleliteralemphasis{\sphinxupquote{complex}}\sphinxstyleliteralemphasis{\sphinxupquote{)}}) \textendash{} The Mermin polynomial \(mn\).

\item {} 
\sphinxstyleliteralstrong{\sphinxupquote{mn\_prime}} (\sphinxstyleliteralemphasis{\sphinxupquote{np.array}}\sphinxstyleliteralemphasis{\sphinxupquote{(}}\sphinxstyleliteralemphasis{\sphinxupquote{complex}}\sphinxstyleliteralemphasis{\sphinxupquote{)}}) \textendash{} The Mermin polynomial \(mn'\).

\item {} 
\sphinxstyleliteralstrong{\sphinxupquote{vector}} (\sphinxstyleliteralemphasis{\sphinxupquote{list}}\sphinxstyleliteralemphasis{\sphinxupquote{(}}\sphinxstyleliteralemphasis{\sphinxupquote{int}}\sphinxstyleliteralemphasis{\sphinxupquote{)}}) \textendash{} The state vector.

\item {} 
\sphinxstyleliteralstrong{\sphinxupquote{bool}} (\sphinxstyleliteralemphasis{\sphinxupquote{type\_of\_mu}}) \textendash{} If False, the classical calculation will be made. 
If not, another method is used.

\end{itemize}

\item[{Returns}] \leavevmode
float \textendash{} The value of the calculation.

\end{description}\end{quote}

\end{fulllineitems}

\index{mu\_file\_saving() (in module mermin\_on\_qiskit.hypergraphstates\_optimization.mermin\_polynomials)@\spxentry{mu\_file\_saving()}\spxextra{in module mermin\_on\_qiskit.hypergraphstates\_optimization.mermin\_polynomials}}

\begin{fulllineitems}
\phantomsection\label{\detokenize{mermin_polynomials-opti:mermin_on_qiskit.hypergraphstates_optimization.mermin_polynomials.mu_file_saving}}\pysiglinewithargsret{\sphinxcode{\sphinxupquote{mermin\_on\_qiskit.hypergraphstates\_optimization.mermin\_polynomials.}}\sphinxbfcode{\sphinxupquote{mu\_file\_saving}}}{\emph{n}, \emph{file\_path}, \emph{first\_mu}, \emph{first\_parameters}, \emph{maximal\_mu}, \emph{maximisation\_parameters}, \emph{type\_of\_mu}}{}
Creates a file to write the various calculation parameters
\begin{quote}\begin{description}
\item[{Parameters}] \leavevmode\begin{itemize}
\item {} 
\sphinxstyleliteralstrong{\sphinxupquote{n}} (\sphinxstyleliteralemphasis{\sphinxupquote{int}}) \textendash{} The number of qubits.

\item {} 
\sphinxstyleliteralstrong{\sphinxupquote{file\_path}} (\sphinxstyleliteralemphasis{\sphinxupquote{string}}) \textendash{} The path where the file is to be saved

\item {} 
\sphinxstyleliteralstrong{\sphinxupquote{first\_mu}} (\sphinxstyleliteralemphasis{\sphinxupquote{float}}) \textendash{} The value of the calculation of the vector with the 
mermin polynomial

\item {} 
\sphinxstyleliteralstrong{\sphinxupquote{first\_parameters}} (\sphinxstyleliteralemphasis{\sphinxupquote{np.array}}\sphinxstyleliteralemphasis{\sphinxupquote{(}}\sphinxstyleliteralemphasis{\sphinxupquote{list}}\sphinxstyleliteralemphasis{\sphinxupquote{(}}\sphinxstyleliteralemphasis{\sphinxupquote{float}}\sphinxstyleliteralemphasis{\sphinxupquote{)}}\sphinxstyleliteralemphasis{\sphinxupquote{)}}) \textendash{} The first coefficients of Mu 
calculation

\item {} 
\sphinxstyleliteralstrong{\sphinxupquote{maximal\_mu}} (\sphinxstyleliteralemphasis{\sphinxupquote{float}}) \textendash{} The value of the maximal Mu calculated

\end{itemize}

\end{description}\end{quote}

\end{fulllineitems}

\index{new\_coefficients\_generation() (in module mermin\_on\_qiskit.hypergraphstates\_optimization.mermin\_polynomials)@\spxentry{new\_coefficients\_generation()}\spxextra{in module mermin\_on\_qiskit.hypergraphstates\_optimization.mermin\_polynomials}}

\begin{fulllineitems}
\phantomsection\label{\detokenize{mermin_polynomials-opti:mermin_on_qiskit.hypergraphstates_optimization.mermin_polynomials.new_coefficients_generation}}\pysiglinewithargsret{\sphinxcode{\sphinxupquote{mermin\_on\_qiskit.hypergraphstates\_optimization.mermin\_polynomials.}}\sphinxbfcode{\sphinxupquote{new\_coefficients\_generation}}}{\emph{n}, \emph{old\_coefficients}, \emph{alpha}}{}
Random generation of new parameters for MU maximization
\begin{quote}\begin{description}
\item[{Parameters}] \leavevmode\begin{itemize}
\item {} 
\sphinxstyleliteralstrong{\sphinxupquote{n}} (\sphinxstyleliteralemphasis{\sphinxupquote{int}}) \textendash{} The number of qubits

\item {} 
\sphinxstyleliteralstrong{\sphinxupquote{old\_coefficients}} (\sphinxstyleliteralemphasis{\sphinxupquote{np.array}}\sphinxstyleliteralemphasis{\sphinxupquote{(}}\sphinxstyleliteralemphasis{\sphinxupquote{list}}\sphinxstyleliteralemphasis{\sphinxupquote{(}}\sphinxstyleliteralemphasis{\sphinxupquote{float}}\sphinxstyleliteralemphasis{\sphinxupquote{)}}\sphinxstyleliteralemphasis{\sphinxupquote{)}}) \textendash{} The table of the coefficients 
that didn’t maximize Mu

\item {} 
\sphinxstyleliteralstrong{\sphinxupquote{alpha}} (\sphinxstyleliteralemphasis{\sphinxupquote{int}}) \textendash{} The value of the descent step (used in the random walk 
method)

\end{itemize}

\item[{Returns}] \leavevmode
np.array(list(float)) \textendash{} Table of new coefficients

\end{description}\end{quote}

\end{fulllineitems}

\index{xbest\_calculation() (in module mermin\_on\_qiskit.hypergraphstates\_optimization.mermin\_polynomials)@\spxentry{xbest\_calculation()}\spxextra{in module mermin\_on\_qiskit.hypergraphstates\_optimization.mermin\_polynomials}}

\begin{fulllineitems}
\phantomsection\label{\detokenize{mermin_polynomials-opti:mermin_on_qiskit.hypergraphstates_optimization.mermin_polynomials.xbest_calculation}}\pysiglinewithargsret{\sphinxcode{\sphinxupquote{mermin\_on\_qiskit.hypergraphstates\_optimization.mermin\_polynomials.}}\sphinxbfcode{\sphinxupquote{xbest\_calculation}}}{\emph{n}, \emph{type\_of\_mu}, \emph{alpha}, \emph{alpha\_minimum}, \emph{c\_maximum}, \emph{vector}, \emph{file\_path}, \emph{saving\_file=True}}{}~\begin{description}
\item[{Maximizes Mu. The algorithm used here is called the Random walk method. }] \leavevmode
The principle is simple. We randomly generate first parameters which are 
used to calculate Mu. The first value of Mu is called Mu0. Then, we 
calculate new parameters based on the previous ones and a variable 
called the descent step. With the new parameters, we calculate a new 
value of Mu. If this value is better than the previous one, we keep it 
and continue the researches until a counter is at its maximum value. The 
goal here is to take a big circle of research scope and to reduce it (by
decreasing the decent step) more and more until the maximum value of Mu 
is found.

\end{description}
\begin{quote}\begin{description}
\item[{Parameters}] \leavevmode\begin{itemize}
\item {} 
\sphinxstyleliteralstrong{\sphinxupquote{n}} (\sphinxstyleliteralemphasis{\sphinxupquote{int}}) \textendash{} The number of qubits.

\item {} 
\sphinxstyleliteralstrong{\sphinxupquote{bool}} (\sphinxstyleliteralemphasis{\sphinxupquote{type\_of\_mu}}) \textendash{} If False is specified, the classical calculation 
will be made. If not, another method is used.

\item {} 
\sphinxstyleliteralstrong{\sphinxupquote{alpha}} (\sphinxstyleliteralemphasis{\sphinxupquote{int}}) \textendash{} The value of the descent step.

\item {} 
\sphinxstyleliteralstrong{\sphinxupquote{alpha\_minimum}} (\sphinxstyleliteralemphasis{\sphinxupquote{int}}) \textendash{} The minimum value of the descent step (which is 
the length of the radius).

\item {} 
\sphinxstyleliteralstrong{\sphinxupquote{c\_maximum}} (\sphinxstyleliteralemphasis{\sphinxupquote{int}}) \textendash{} The maximum value of the counter.

\item {} 
\sphinxstyleliteralstrong{\sphinxupquote{vector}} (\sphinxstyleliteralemphasis{\sphinxupquote{list}}\sphinxstyleliteralemphasis{\sphinxupquote{(}}\sphinxstyleliteralemphasis{\sphinxupquote{int}}\sphinxstyleliteralemphasis{\sphinxupquote{)}}) \textendash{} The vector for the calculation of Mu.

\item {} 
\sphinxstyleliteralstrong{\sphinxupquote{file\_path}} (\sphinxstyleliteralemphasis{\sphinxupquote{string}}) \textendash{} The path where the file is to be saved.

\item {} 
\sphinxstyleliteralstrong{\sphinxupquote{saving\_file}} (\sphinxstyleliteralemphasis{\sphinxupquote{boolean}}) \textendash{} If set to True, a file will be created / 
overloaded with the information about the calculation of Mu. If not, 
only the calculations are made.

\end{itemize}

\item[{Returns}] \leavevmode
np.array(list(float)) \textendash{} The array that contains the parameters 
that maximizes Mu.

\end{description}\end{quote}

\end{fulllineitems}



\chapter{Indices and tables}
\label{\detokenize{index:indices-and-tables}}\begin{itemize}
\item {} 
\DUrole{xref,std,std-ref}{genindex}

\item {} 
\DUrole{xref,std,std-ref}{modindex}

\item {} 
\DUrole{xref,std,std-ref}{search}

\end{itemize}


\renewcommand{\indexname}{Python Module Index}
\begin{sphinxtheindex}
\let\bigletter\sphinxstyleindexlettergroup
\bigletter{m}
\item\relax\sphinxstyleindexentry{mermin\_on\_qiskit.basis\_change}\sphinxstyleindexpageref{basis_change:\detokenize{module-mermin_on_qiskit.basis_change}}
\item\relax\sphinxstyleindexentry{mermin\_on\_qiskit.coefficients\_shapes}\sphinxstyleindexpageref{coefficients_shapes:\detokenize{module-mermin_on_qiskit.coefficients_shapes}}
\item\relax\sphinxstyleindexentry{mermin\_on\_qiskit.evaluation}\sphinxstyleindexpageref{evaluation:\detokenize{module-mermin_on_qiskit.evaluation}}
\item\relax\sphinxstyleindexentry{mermin\_on\_qiskit.hypergraphstates}\sphinxstyleindexpageref{hypergraphstates:\detokenize{module-mermin_on_qiskit.hypergraphstates}}
\item\relax\sphinxstyleindexentry{mermin\_on\_qiskit.hypergraphstates\_optimization.hypergraphstates}\sphinxstyleindexpageref{hypergraphstates-opti:\detokenize{module-mermin_on_qiskit.hypergraphstates_optimization.hypergraphstates}}
\item\relax\sphinxstyleindexentry{mermin\_on\_qiskit.hypergraphstates\_optimization.mermin\_polynomials}\sphinxstyleindexpageref{mermin_polynomials-opti:\detokenize{module-mermin_on_qiskit.hypergraphstates_optimization.mermin_polynomials}}
\item\relax\sphinxstyleindexentry{mermin\_on\_qiskit.QFT}\sphinxstyleindexpageref{QFT:\detokenize{module-mermin_on_qiskit.QFT}}
\item\relax\sphinxstyleindexentry{mermin\_on\_qiskit.run}\sphinxstyleindexpageref{run:\detokenize{module-mermin_on_qiskit.run}}
\end{sphinxtheindex}

\renewcommand{\indexname}{Index}
\printindex
\end{document}